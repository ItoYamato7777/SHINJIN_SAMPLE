\documentclass[twocolumn, a4paper]{Zemiresume}
\usepackage[dvipdfmx]{graphicx}
\usepackage{graphicx}
\usepackage{amsmath}
\usepackage{txfonts}

%追加したパッケージ
\usepackage{siunitx}
\usepackage{xcolor}
\usepackage{url}
\usepackage{silence}
\WarningFilter{caption}{Unknown document class (or package)} % captionに関する警告を無視
\usepackage{subcaption}

\title{"遊び"が"競技"になる時 \\ - 格闘ゲームEスポーツの世界 -}
\date{2024年 8月 22日}
\author{伊藤 大翔}
\headtitle{2024年度 工藤・木村研究室 ゼミ合宿}

\begin{document}
\maketitle

%----------------------------------------------------------------------------------------------------------
\section{はじめに}
筆者は中学生の時にテレビで見た大会がきっかけで,格闘ゲーム「ストリートファイター」に魅了され,以来そのプロシーンを追い続けている.
当時は単なる”ゲーム”であったものが,現在では数万人の観客を熱狂させ,億単位の賞金が動く「Eスポーツ」という巨大な産業へと変貌を遂げた.
本稿では,格闘ゲームの金字塔であるストリートファイターを題材に,Eスポーツが持つ競技としての面白さと,その最前線で戦う選手たちが放つ”人間ドラマ”の魅力について述べる.

%----------------------------------------------------------------------------------------------------------
\section{Eスポーツとは?}
Eスポーツ(Esports)は「エレクトロニック・スポーツ」の略であり,ビデオゲームを競技として捉える文化の総称である.
そこでは,単なるゲームの腕前だけでなく,高度な戦略性,0.1秒を争う反射神経,そして極限状態での判断力といった,従来のスポーツにも通じる強靭な精神力が求められる.

その熱狂は市場規模にも表れている.世界のEスポーツ市場は年々拡大を続けており,2024年には約18億ドルに達すると予測されている\cite{cite:newzoo2024}.
図\ref{fig:evo_venue}に示すように,大規模なオフライン大会では数万人の観客が会場を埋め尽くし,その熱気は他のプロスポーツにも引けを取らない.

\begin{figure}[t]
  \centering
  \includegraphics[width=\columnwidth]{img/evo_venue.png}
  % img/evo_venue.png という名前で大会会場の熱気が伝わる画像を保存してください
  \caption{大規模Eスポーツ大会の会場風景}\label{fig:evo_venue}
\end{figure}

%----------------------------------------------------------------------------------------------------------
\section{ストリートファイターとEスポーツの歩み}
\subsection{黎明期(1990年代〜)}
ストリートファイターの競技シーンの原点は,1990年代のゲームセンター文化にある.プレイヤー達は自然発生的にコミュニティを形成し,対戦を通じて技術を磨き合った.

\subsection{転換期(2008年〜)}
オンライン対戦機能を備えた『ストリートファイターIV』の登場は,世界中のプレイヤーを繋ぎ,競技シーンを爆発的に拡大させた.これが現代Eスポーツの礎を築いたと言える.

\subsection{確立期(2014年〜)}
ゲームメーカー主導による公式世界大会「CAPCOM Pro Tour」が設立され,プロライセンス制度も導入された.これにより,”遊び”は公式な”職業”となり,多くのプロゲーマーが誕生した.

%----------------------------------------------------------------------------------------------------------
\section{勝負の世界に生きる者たち}
Eスポーツの魅力は,その競技性だけではない.異なる背景と哲学を持つ選手たちが織りなす人間ドラマこそが,観る者を惹きつける最大の要因である.ここでは筆者が特に注目する3名の選手を紹介する.

\subsection{"The Beast" 梅原 大吾(ウメハラ)}
「世界で最も長く勝ち続けるプロゲーマー」としてギネス世界記録にも認定される”リビングレジェンド”である\cite{cite:guinness_umehara}.
2004年の大会で見せた「背水の逆転劇」は,絶体絶命の状況から相手の必殺技17回を全て捌き切って勝利した奇跡的なプレイであり,Eスポーツ史に残る伝説として語り継がれている.彼の勝利への飽くなき探求心と独自の勝負哲学は,著書『勝ち続ける意志力』\cite{cite:umehara_book}にも記されており,ゲームの枠を超えて多くの人々に影響を与えている.

\begin{figure}[b]
  \centering
  \includegraphics[width=0.8\columnwidth]{img/umehara.jpg}
  % img/umehara.jpg という名前で梅原選手の画像を保存してください
  \caption{梅原 大吾選手}\label{fig:umehara}
\end{figure}

\subsection{"東大卒プロゲーマー" ときど}
東京大学大学院出身という異色の経歴を持つ知性派プレイヤー.対戦相手を研究者のように徹底的に分析し,あらゆる努力を惜しまないスタイルから”努力の天才”と称される.
彼の存在は,Eスポーツが反射神経や才能だけでなく,情報戦や戦略といった”知性”が勝敗を分ける奥深い競技であることを象徴している.

\subsection{"新世代の絶対王者" カワノ}
近年,数々の大舞台を制し,梅原・ときどといったレジェンド達の前に立ちはだかる新世代の筆頭.人間離れした反応速度と対応力を武器に,上の世代が築いたセオリーを純粋な”ゲーム力”で破壊していく姿は圧巻である.どのスポーツにも存在する「世代交代」というドラマを体現する存在として,シーンを牽引している.

%----------------------------------------------------------------------------------------------------------
\section{まとめ}
本稿では,ストリートファイターを例にEスポーツの世界を紹介した.
Eスポーツの面白さは,華やかな大会やスーパープレイだけではない.紹介した3名のように,異なる才能,異なる哲学を持つ選手たちが,それぞれのやり方で「好き」を突き詰め,人生を懸けて頂点を目指す姿こそが,最大の魅力である.
そのひたむきな姿は,我々が研究や目標に向き合う姿勢にも,多くの示唆を与えてくれるのではないだろうか.

%----------------------------------------------------------------------------------------------------------
{\small
\bibliographystyle{jplain}
\bibliography{template}
}
\end{{document}